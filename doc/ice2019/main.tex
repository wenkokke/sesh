\documentclass[copyright,creativecommons]{eptcs}
\providecommand{\event}{ICE 2019} % Name of the event you are submitting to
\usepackage{textcomp}
\usepackage{textgreek}
\usepackage{doi}
\usepackage[all]{foreign}

% compact enums
\usepackage{enumitem}
\setlist[enumerate]{nosep}

% red squiggly
\usepackage{ulem}
\normalem%
\makeatletter
\def\uwave{\bgroup \markoverwith{\lower3.5\p@\hbox{\sixly \textcolor{red}{\char58}}}\ULon}
\font\sixly=lasy6 % does not re-load if already loaded, so no memory problem.
\makeatother

% load listings-rust
\usepackage{inconsolata}
\usepackage{listings}
\usepackage{listings-rust}
\lstset{
  language=Rust,
  style=colouredRust,
  basicstyle=\ttfamily\small,
  columns=fullflexible,
  xleftmargin=\parindent,
  moredelim=[is][\uwave]{~}{~}
}

% load cleverref
\usepackage{cleveref}

\title{Rusty Variation\\ Deadlock-free Sessions with Failure in Rust}
\author{Wen Kokke
  \institute{LFCS, University of Edinburgh, 10 Crichton St, EH8 9AB, Edinburgh, United Kingdom}
  \email{wen.kokke@ed.ac.uk}
}
\def\titlerunning{Rusty Variation}
\def\authorrunning{Wen Kokke}

\begin{document}

\maketitle

\begin{abstract}
  Rusty Variation (RV) is a library for session-typed communication in Rust which offers strong compile-time correctness guarantees. Programs written using RV are guaranteed to respect a specified protocol, and are guaranteed to be free from deadlocks and races.
\end{abstract}

\section{Introduction}\label{sec:introduction}
In concurrent programming, processes commonly exchange information by sending messages on shared channels. However, this communication often does not follow a specified protocol, and if it does, it the correctness of the implementation with respect to that protocol is rarely checked.

Session types are a substructural typing discipline, introduced by Honda~\cite{honda1993}, which describe communication protocols as \emph{channel types}, capturing both the type and the order of messages. Session type systems, then, ensure that programs correctly implement specified protocols.

Rust is an immensely popular systems programming language.\footnote{\url{https://insights.stackoverflow.com/survey/2019/}} It aims to combine efficiency with abstraction and thread and memory safety. However, out of the box, it does not offer a way to specify protocols and verify communication.

In this paper, we present Rusty Variation\footnote{\url{https://github.com/wenkokke/rusty-variation/}} (RV), a library for session-typed communication in Rust which allows users to specify and check the adherence to communication protocols, in addition to offering strong correctness guarantees, such as freedom from deadlocks and unwanted races, while accounting for the possibility of failure. It is these strong guarantees of correctness, together with out account of failure, which sets us apart from previous work~\cite{jespersen2015}. We assume some familiarity with Rust. For an introduction to Rust, we refer the reader to \textit{The Rust Programming Language}.\footnote{\url{https://doc.rust-lang.org/nightly/book/}}

We base our work on Exceptional GV by Fowler \etal~\cite[EGV]{fowler2019}. Exceptional GV is a descendant of Good Variation~\cite[GV]{wadler2012} which adds support for exceptions. Crucially, EGV has an affine type system, and permits the explicit cancellation of sessions. EGV guarantees session fidelity, global progress, freedom from deadlocks and livelocks, confluence, and termination. That is to say, there are no locks or race conditions, and all programs eventually halt. For a discussion of EGV and its metatheory, we refer the reader to the work by Fowler \etal~\cite{fowler2019}.

Rusty Variation is an implementation of Exceptional GV (EGV) in Rust. We claim that RV is a faithful implementation of EGV, preserving much of EGV's metatheory, meaning RV guarantees session fidelity, global progress, confluence, and freedom from deadlocks and unwanted races. However, we depart from EGV in two ways:
First, EGV relies upon exceptions and handlers. Rust, however, rejects exceptions in favour of monadic errors, \ie, wrapping possibly failing results in error types such as \lstinline{Option<T>} and \lstinline{Result<T,E>}.
Second, The core language of EGV does not include general recursion. Rust does. In the presence of non-termination, we cannot guarantee livelock freedom, or---well---termination.

The paper proceeds as follows. In~\cref{sec:contribution}, we describe the programming interface offered by Rusty Variation, and provide several example programs. In~\cref{sec:related-work}, we compare Rusty Variation to the related work by Laumann \etal~\cite{jespersen2015}. Finally, in~\cref{sec:future-work}, we describe future work on verifying the correctness guarantees offered by RV.

\section{Contribution: Rusty Variation}\label{sec:contribution}

\subsection{Types, sessions, and duality}
There are three basic types which we use to describe session protocols---or \emph{session types}: \lstinline{Send}, \lstinline{Recv}, and \lstinline{End}. We refer to values of these types as \emph{session endpoints} or, when referring to both endpoints, \emph{sessions}. Session endpoints wrap primitive channels---see~\cref{sec:implementation}. Any type built using these primitive types is a valid session type:
\begin{lstlisting}
pub struct Send<T, S: Session> { ... }
pub struct Recv<T, S: Session> { ... }
pub struct End { ... }
\end{lstlisting}
Session types are public, but their member functions are not. Therefore, users cannot construct their own misbehaving sessions. The continuation of a session must itself be a session. This is enforced by the \lstinline{Session} constraint. In addition to this ``kinding'' of session types, the \lstinline{Session} trait also requires any session type to implement duality and session generation:
\begin{lstlisting}
pub trait Session {
  type Dual: Session<Dual=Self>;
  fn new() -> (Self, Self::Dual);
}
\end{lstlisting}
Duality works as expected: \lstinline{Send<T, S>} is dual to \lstinline{Recv<T, S::Dual>}---and vice versa---and \lstinline{End} is self-dual. The constraint \lstinline{Session<Dual=Self>} enforces that the dual of a session type is a session type, and that \lstinline{S::Dual::Dual} is equal to \lstinline{S} (involutivity).

The \lstinline{new} function generates a new session-typed channel and returns two \emph{dual} channel endpoints. Unlike the rest of the functions discussed in this section, \lstinline{new} is \emph{not} part of the public interface---see \cref{sec:limitations}.

Using these types, we can define the types of servers which offer squaring or negation as follows, where \lstinline{i32} is the type of 32-bit integers:
\begin{lstlisting}
type SqrSrv = Recv<i32, Send<i32, End>>;
type NegSrv = Recv<i32, Send<i32, End>>;
\end{lstlisting}
The client types are obtained by duality.

\subsection{Send, receive, close, and fork}\label{sec:basics}
The \lstinline{send} and \lstinline{recv} functions send and receive values in sessions. Sending is non-blocking, and never fails. Sending a value on a cancelled channel is equivalent to cancelling that value---see \cref{sec:cancellation}. Receiving is blocking, and fails if the channel is cancelled.
\begin{lstlisting}
pub fn send<T, S: Session>(x: T, s: Send<T, S>) -> S
pub fn recv<T, S: Session>(s: Recv<T, S) -> Option<(T, S)>
\end{lstlisting}
The \lstinline{close} function closes a \emph{completed} session, \ie of type \lstinline{End}. Closing is \emph{synchronous}, meaning that \lstinline{close} blocks until the dual end point is also closed. Closing a cancelled channel fails.
\begin{lstlisting}
pub fn close(s: End) -> Option<()>
\end{lstlisting}
The \lstinline{fork} function creates a new session, and spawns a new thread. The new thread runs the first argument of \lstinline{fork}---a function---which receives one endpoint, and \lstinline{fork} returns the dual endpoint. It has the following type---where \lstinline{FnOnce(S) -> Option<()>} is the type of functions from \lstinline{S} to \lstinline{Option<()>} which can be invoked only once:
\begin{lstlisting}
pub fn fork<S: Session>(p: P) -> S::Dual
where
  P: FnOnce(S) -> Option<()>,
\end{lstlisting}
The function passed to \lstinline{fork} is allowed to fail, hence the \lstinline{Option}-type. If it fails, the child thread panics silently, and all open sessions in the child thread are cancelled.

Now that we have covered the basics, let us consider an example program. The following program forks off a child process which sends a ``ping'', then waits to receive that ping and returns:
\begin{lstlisting}
let s = fork(move |s: Send<(), End>| {
  let s = send((), s);
  close(s)
});
let ((), s) = recv(s)?;
close(s)
\end{lstlisting}
The \lstinline{let x = M?; N} construct is Rust's ``monadic bind'' notation for programs which may return errors. If \lstinline{recv(s)} succeeds, the \lstinline{?}-operator unpacks the \lstinline{Option}. If \lstinline{recv(s)} fails, the \lstinline{?}-operator short-circuits, skips the rest of the statements, and returns the error.

\subsection{Linearity and cancellation}\label{sec:cancellation}
Session types often require linearity---the protocol for a negation server in~\cref{sec:basics} specifies that the server receives \emph{one} integer, and then sends \emph{one} integer back. For such a server to, \eg send back multiple integers would be, at best, confusing, and at worst, cause an error. The following two erroneous programs violate the linearity condition. The program on the left reuses the channel s, and the program on the right drops it without completing the session:

\vspace{-0.5\baselineskip}
\begin{minipage}[t]{0.5\linewidth}
\begin{lstlisting}
let s = fork(move |s: Send<(), End>| {
  let s1 = send((), s);
  close(s1)?;
  let s2 = send((), ~s~); // reuse s
  close(s2)
});
let ((), s) = recv(s)?;
close(s)
\end{lstlisting}
\end{minipage}%
\begin{minipage}[t]{0.5\linewidth}
\begin{lstlisting}
let s = fork(move |s: Send<(), End>| {
  
  // unintentionally
  // left blank.

});
let ((), s) = recv(s)?;
close(s)
\end{lstlisting}
\end{minipage}

\noindent
Rust is an affine language, meaning that any value can be used \emph{at most once}. Hence, our first example doesn't compile: Rust complains about the second usage of \lstinline{s}, which attempts to use \lstinline{s} after transferring ownership of the value to \lstinline{send}.

The second program is trickier, as Rust's type system doesn't help us here. Arguably, however, that is for the best. The idea of \emph{linear} session types ignores the reality that sessions may be dropped at any time, for any number of reasons: a process may panic, dropping all sessions in scope, or---when communicating over the network---the connection might fail. In~\cref{sec:introduction}, we mentioned that EGV has an affine type system, and permits the explicit cancellation of sessions. Following EGV, Rusty Variation also permits the cancellation of sessions. However, instead of having cancellation be an explicit primitive, we implement it implicitly for any dropped value, using Rust's destructors\footnote{\url{https://doc.rust-lang.org/nomicon/destructors.html}}.

When an initialised value in Rust goes out of scope, Rust calls its destructor, the \lstinline{drop} method. This method can be customised by implementing the \lstinline{Drop} trait. For instance, if we use the implementation of \lstinline{Drop} shown on the left, running the code on the right would print ``End Dropped!'' twice, once from the child thread, and once from the main thread.

\vspace{-0.5\baselineskip}
\begin{minipage}[t]{0.5\linewidth}
\begin{lstlisting}
impl Drop for End {
  fn drop(&mut self) {
    println!("End Dropped!")
  }
}
\end{lstlisting}
\end{minipage}%
\begin{minipage}[t]{0.5\linewidth}
\begin{lstlisting}
let _s = fork(move |_s: End| {
  // intentionally left blank.
});
// intentionally left blank.
\end{lstlisting}
\end{minipage}

\noindent
To mimic the cancellation behaviour in EGV, we count the number of references to a session, starting at \emph{two} when a session is created using \lstinline{new}. Each time a session's destructor is called, we decrement this counter. If the counter is \emph{one}, the session is marked as \emph{disconnected}. Sending on a disconnected session drops the value being sent. Receiving on or closing disconnected sessions returns an error, \ie \lstinline{None}. This ensures that no process is left waiting for a message that will never come. If the counter reaches \emph{zero}, the memory for the session is deallocated.

As it turns out, this is exactly the behaviour for the channels from \lstinline{crossbeam_channel}, which we use in our implementation, and when a session is dropped, the channel it wraps is dropped recusively. Hence, we inherit the correct behaviour of dropping sessions from \lstinline{crossbeam_channel}.

However, there is one major downside to replacing explicit cancellation with implicit cancellation: it is no longer possible to tell whether the programmer wanted to cancel a session or simply forgot to complete the session. We do two things to remedy this. First, the definitions of \lstinline{Send}, \lstinline{Recv}, and \lstinline{End} are annotated with \lstinline{#[must_use]}. This causes Rust to emit a warning whenever a session is dropped. Second, we provide an explicit \lstinline{cancel} function, though its implementation is simply to drop its argument:
\begin{lstlisting}
  pub fn cancel<T>(_x: T) -> () { /* intentionally left blank. */ }
\end{lstlisting}

The following two programs illustrate our solution. The program on the left forgets to use \lstinline{s}, and Rust emits a warning, complaining that \lstinline{s} isn't used. The program on the right passes the session to \lstinline{cancel}, explicitly marking the drop, and hence compiles without any warnings. Both programs have the same semantics, though: the program forks off a process which is \emph{expected} to send a ``ping''. Instead the forked process cancels the session---be it implicitly or explicitly. Hence, the call to \lstinline{recv} fails, the \lstinline{?}-operator short-circuits, and the whole program returns \lstinline{None}:

\vspace{-0.5\baselineskip}
\begin{minipage}[t]{0.5\linewidth}
\begin{lstlisting}
let s = fork(move |s: Send<(), End>| {
  // shouldn't we be using s?
  Some(())
});
let (z, s) = recv(s)?;
close(s)
\end{lstlisting}
\end{minipage}%
\begin{minipage}[t]{0.5\linewidth}
\begin{lstlisting}
let s = fork(move |s: Send<(), End>| {
  cancel(s);
  Some(())
});
let (z, s) = recv(s)?;
close(s)
\end{lstlisting}
\end{minipage}%


\subsection{Branching with offer! and choose!}\label{sec:choice}
The \lstinline{choose_left}, \lstinline{choose_right}, and \lstinline{offer_either} functions let processes offer or make a \emph{binary} choice. These are \emph{derived} constructs~\cite{honda1993,dardha2017}---they are implemented using only the other functions in our library. They send or receive a value of the sum type, \lstinline{Either<S1, S2>}, where \lstinline{S1} and \lstinline{S2} are the possible continuations of the session.
\begin{lstlisting}
type Choose<S1, S2> = Send<Either<S1, S2>, End>;
type Offer<S1, S2> = Recv<Either<S1, S2>, End>;

pub fn choose_left<S1: Session, S2: Session>(s: Choose<S1, S2>) -> S1
pub fn choose_right<S1: Session, S2: Session>(s: Choose<S1, S2>) -> S2
pub fn offer_either<S1: Session, S2: Session, P1, P2, R>(
       s: Offer<S1, S2>, p1: P1, p2: P2) -> Option<R>
where
  P1: FnOnce(S1) -> Option<R>,
  P2: FnOnce(S2) -> Option<R>,
\end{lstlisting}
There is a subtlety in their implementation. The \lstinline{choose_left} function creates a new session of type \lstinline{S1}. It wraps one endpoint in \lstinline{Either::Left}, sends it over the original session \lstinline{s}, and returns the other endpoint. However, this leaves it with the continuation of the original session, of type \lstinline{End}. Closing this endpoint synchronously, using \lstinline{close}, may fail. However, \lstinline{choose_left} \emph{sends}, and for consistency with \lstinline{send}, we would like it to always succeed. Therefore, we opt to \emph{cancel} the continuation instead:
\begin{lstlisting}
pub fn choose_left<S1: Session, S2: Session>(s: Choose<S1, S2>) -> S1 {
    let (here, there) = S1::new();
    let s = send(Either::Left(there), s);
    cancel(s);
    here
} 
\end{lstlisting}
Cancelling the endpoint is harmless---as long as the other processes is prepared, and uses the corresponding function \lstinline{offer_either}, this acts as an \emph{asynchronous} alternative to close. Pairing \lstinline{choose_left} with a manually implemented version of \lstinline{offer_either} which attempts to \emph{close} the continuation instead of cancelling it results in a runtime error.

The \lstinline{choose_left}, \lstinline{choose_right}, and \lstinline{offer_either} functions suffice to encode choice, they are not useful in practice---a protocol with a four-way choice would require three nested \lstinline{Offer} constructs, and several calls to \lstinline{offer_either}. Code written in this style rapidly becomes unreadable. For this reason, we also define two macros, \lstinline{choose!} and \lstinline{offer!}, which generalise binary choice using \lstinline{Either} to a choice on \emph{any} \lstinline{enum}, as long as each case wraps a session. This allows processes to offer and make choices between many \emph{labelled} branches, and allows us to define the type for a calculator server as:
\begin{lstlisting}
enum CalcOp { Sqr(SqrSrv), Neg(NegSrv) }
type CalcSrv = Recv<CalcOp, End>;
type CalcCli = CalcSrv::Dual;
\end{lstlisting}
Using the \lstinline{choose!} and \lstinline{offer!} macros, we can implement a calculator server and client. The \lstinline{main} function below forks off a calculator server, which offers the choice between squaring and negation. The client then selects squaring, sends the number four, and prints the result:

\begin{minipage}[t]{0.5\linewidth}
\begin{lstlisting}
fn server(s: CalcSrv) -> Option<()> {
  offer!(s, {
    CalcOp::Sqr(s) => {
      let (x, s) = recv(s)?;
      let s = send(x * x, s);
      close(s)
    },
    CalcOp::Neg(s) => {
      let (x, s) = recv(s)?;
      let s = send(-x, s);
      close(s)
    },
  })
}
\end{lstlisting}
\end{minipage}%
\begin{minipage}[t]{0.5\linewidth}
\begin{lstlisting}
fn client(s: CalcCli) -> Option<i32> {
  let s = choose!(CalcOp::Sqr, s);
  let s = send(4, s);
  let (z, s) = recv(s)?;
  close(s)?;
  Some(z)
}

// main() prints Some(16)
fn main() {
  let s = fork(server);
  let z = client(s);
  println!("{}", z);
}
\end{lstlisting}
\end{minipage}

\subsection{Selection}\label{sec:selection}
Selection is a mechanism which allows processes to block on a list of actions, waiting until the first of them fires. Writing programs which use selection can be tricky and error prone, even using libraries specifically built to help with selection. For instance, the following is an excerpt from the documentation of \lstinline{crossbeam_channel::Select}\footnote{\url{https://docs.rs/crossbeam-channel/0.1.2/crossbeam_channel/struct.Select.html}}, a construct which implements a selection mechanism:
\begin{quotation}
  \noindent
  Rules which must be respected in order for selection to work properly:
  \begin{enumerate}
  \item Before each selection, a fresh Select must be created.
  \item Selection cases must be repeatedly probed in a loop.
  \item If a selection case fires, the loop must be broken without probing cases any further.
  \item In each iteration of the loop, the same set of cases must be probed in the same order.
  \item No selection case may be repeated.
  \item No two cases may operate on the same end (receiving or sending) of the same channel.
  \item There must be at least one send, or at least one recv case.
  \end{enumerate}
\end{quotation}
None of these rules are enforced by Rust---instead, all of them must be checked by the programmer---and bugs in implementing selection can be hard to find.

RV offers an alternative, the \lstinline{select_mut} function. This function operates on a vector of \emph{receiving} endpoints, returning the result of the first receive operation which completes, and removing the corresponding session from the vector. We restrict ourselves to the receiving endpoints, as RV's \lstinline{send} operation is non-blocking, hence selecting on sending endpoints doesn't make any sense.
\begin{lstlisting}
pub fn select_mut<T, S: Session>(rs: &mut Vec<Recv<T, S>>) -> Option<(T, S)>
\end{lstlisting}

The \lstinline{select_mut} function is implemented on top of \lstinline{crossbeam_channel::Select}, in a way which respects the rules listed above. In addition, RV offers an immutable variant of \lstinline{select_mut}, named \lstinline{select}, which returns the remainder of the endpoints along the with result of the selection. Using \lstinline{select_mut}, we can write the following program, which forks off ten threads, each sending an index between 1 and 10, and then repeatedly selects on the receiving endpoints, receiving and printing an index, printing the numbers 1 to 10 in a random order:

\vspace{-0.5\baselineskip}
\begin{minipage}[t]{0.5\linewidth}
\begin{lstlisting}
let mut rs = Vec::new();
for i in 0..10 {
  let s = fork(move |s: Send<i32, End>| {
    let s = send(i, s);
    close(s)
  });
  rs.push(s);
}
\end{lstlisting}
\end{minipage}
\begin{minipage}[t]{0.5\linewidth}
\begin{lstlisting}
loop {
  if rs.is_empty() {
    break Ok(());
  }
  let (i, r) = select_mut(&mut rs)?;
  println!("{}", i);
  close(r)
}
\end{lstlisting}
\end{minipage}

The \lstinline{select_mut} and \lstinline{select} functions are homogeneous: they only select over vectors of endpoints with \emph{the same type}. However, this does not rule out heterogeneous uses of selection, where the operations are of different types. It is possible to perform heterogeneous selection by defining an enumeration, in the style of the \lstinline{CalcOp} type mentioned in~\cref{sec:choice}.

TODO: THIS IS WHERE I STOPPED

\subsection{Implementation}\label{sec:implementation}
We use the asynchronous channels from \lstinline{crossbeam_channel}. These are ideal, as they already implement the correct cancellation behaviour. Channels are typed, and can only transfer values of a single type. Therefore, we use one-shot channels, and encode sessions following Scalas and Yoshida, and Padovani~\cite{scalas2016,padovani2017}: each \lstinline{send} creates a new channel for the continuation of the session, and sends that along with the value, and each \lstinline{recv} receives a value together with a channel on which to continue the session. Fortunately, the channels in \lstinline{crossbeam_channel} are highly optimised for one-shot usage.

\subsection{What If I Want to Be Evil?}\label{sec:limitations}
There are three ways in which you can make code written using Rusty Variation lock. The first is by looping, which we discussed earlier. The other two are explicit, and surprisingly, \lstinline{unsafe} is not one of them.

The biggest issue lies in the fact that Rust's semantics do not guarantee that destructors are actually run. This is important for manually managed memory, \eg doubly linked lists. For this reason, it is always possible to ``forget'' a value, using \lstinline{std::mem::forget}, without running its destructors.

Secondly, it is not possible---at the time of writing---to export a trait without exporting all its members. Therefore, any user can use the \lstinline{new} function, and doing so, it is trivial to write a program which deadlocks.

\begin{minipage}[t]{0.5\linewidth}
\begin{lstlisting}
// Deadlock using std::mem::forget
let s = fork(move |s: Send<(), End>| {
  std::mem::forget(s);
  Some(())
});
let ((), s) = recv(s)?;
close(s)
\end{lstlisting}
\end{minipage}%
\begin{minipage}[t]{0.5\linewidth}
\begin{lstlisting}
// Deadlock using new and spawn
let (s1, r1) = Send<Void, End>::new();
let (s2, r2) = Send<Void, End>::new();
std::thread::spawn(move || {
  let (v, r1) = recv(r1)?;
  close(r1)?;
  let s2 = send(v, s2);
  close(s2)
});
let (v, r2) = recv(r2)?;
close(r2)?;
let s1 = send(v, s1);
close(s1)
\end{lstlisting}
\end{minipage}

\section{Related Work}\label{sec:related-work}
Laumann \etal~\cite{jespersen2015} implement session types in Rust, based on the session-typed functional language LAST by Gay and Vasconcelos~\cite{gay2009}. Their implementation guarantees that programs correctly implement specified communication protocols with some caveats. LAST depends upon linear types, meaning that sessions have to be used exactly once. Rust only offers affine types, meaning that  values---and, particularly relevant here, sessions---can be dropped. The implementation of LAST in Rust, therefore, guarantees session fidelity under the assumption that channels are \emph{never dropped}. In some cases, the compiler can be made to issue a warning at compile-time when a channel is dropped, for instance through the \lstinline{#[must_use]} annotation. However, in general, this is not something the compiler can guarantee. Furthermore, linear session types ignore the reality that channels may be dropped at any time due to, \eg a thread panicking.

We embrace the affine, and base our work on Exceptional GV by Fowler \etal~\cite[EGV]{fowler2019}. Exceptional GV is a descendant of Good Variation~\cite[GV]{wadler2012}, which itself is a descendant of LAST. Crucially, EGV has an affine type system, and allows for the explicit cancellation of sessions. Furthermore, EGV offers a stronger metatheory than LAST. In addition to session fidelity, EGV also guarantees global progress, freedom from deadlocks and livelocks, confluence, and termination. That is to say, there are no locks or race conditions, and all programs eventually halt. For a discussion of EGV and its metatheory, we refer the reader to the work by Fowler \etal~\cite{fowler2019}.

\vspace{-0.5\baselineskip}
\begin{minipage}[t]{0.5\linewidth}
\begin{lstlisting}
// Rusty Variation
// main() returns None
fn server(_s: Recv<(), End>) {
  // unintentionally left blank.
}
fn client(s: Send<(), End>) {
  close(send((), s))
}
fn main() -> Option<()> {
  client(fork(server))
}
\end{lstlisting}
\end{minipage}%
\begin{minipage}[t]{0.5\linewidth}
\begin{lstlisting}
// Laumann et al.
// main() causes segfault
fn server(_c: Chan<(), Recv<(), Eps>>) {
  // unintentionally left blank.
}
fn client(c: Chan<(), Send<(), Eps>>) {
  c.send(()).close();
}
fn main() -> () {
  connect(server, client);
}
\end{lstlisting}
\end{minipage}%

\vspace{-0.5\baselineskip}
\begin{minipage}[t]{0.5\linewidth}
\begin{lstlisting}
// Laumann et al.
// The order in the type doesn't match
// the order in the code. This compiles,
// not because the cases are associated
// correctly, but because the labels are
// not checked.

type SqrSrv  = Recv<i32, Send<i32, Eps>>;
type NegSrv  = Recv<i32, Send<i32, Eps>>;
type CalcSrv = Offer<SqrSrv, NegSrv>;

fn server(c: Chan<(), CalcSrv>) {
  offer!{c,
    NEG => {
      let (c, n) = c.recv();
      let c = c.send(-n);
      c.close();
    },
    SQR => {
      let (c, n) = c.recv();
      let c = c.send(n * n);
      c.close();
    }
  }
}
\end{lstlisting}
\end{minipage}%
\begin{minipage}[t]{0.5\linewidth}
\begin{lstlisting}

// The client selects the first option,
// assuming it represents squaring...
// Unfortunately, it represents negation.



type SqrCli  = Send<i32, Recv<i32, Eps>>;
type NegCli  = Send<i32, Recv<i32, Eps>>;
type CalcCli = Choose<SqrCli, NegCli>;

fn client(c: Chan<(), CalcCli>) {
  let c = c.sel1(); // select SqrCli
  let c = c.send(42);
  let (c, n) = c.recv();
  println!("{}", n);
  c.close();
}

fn main() {
  connect(server, client); // prints -42
}
\end{lstlisting}
\end{minipage}


\subsection{Future Work}\label{sec:future-work}
In \cref{sec:contribution}, we claim that RV preserves the metatheory of EGV, and while a full discussion of the testing and formal verification efforts we have undertaken to ensure the correctness of RV is out of the scope of this paper, it would be remiss of us not substantiate this claim.

Firstly, RV is unit tested. We test various use-cases from EGV, from basic operations and their interactions with cancellation, to delegation and recursive session types. This test suite has caught several errors in the implementation. However, unit tests are error prone, and it is hard (if not impossible) to obtain a set of tests which cover all use-cases of a library. Unfortunately, tools such as RustBelt~\cite{jung2017} and Rust Distilled~\cite{weiss2018} are not yet mature enough to allow us to \emph{prove} the correctness of RV.

In the absence of a mature formal semantics for Rust, we are designing a formal language, RV (or ``formal RV''), which matches, as close as possible, the semantics of our Rust library. It is a concurrent lambda calculus with heap-based shared-memory semantics. We plan to prove that formal RV guarantees preservation, progress, termination, confluence and the diamond property, and deadlock and livelock freedom. Furthermore, we plan to show that there exists a translation from EGV to formal RV, based on the monadic reflection of exceptions into the \lstinline{Option} monad~\cite{filinski1994} and the translation of channel-based communication to shared memory, which is in operational correspondence with EGV. This would show that formal RV is a faithful implementation of EGV.

Lastly, we plan to ensure the correspondence between formal RV and the implementation of RV, using property-based testing. We plan to generate a series of random formal RV terms, using Neat~\cite{claessen2015}, and check if their behaviour, when run as Rust programs, corresponds to the behaviour of the term in formal RV.

\bibliographystyle{eptcs}
\bibliography{main}

\end{document}