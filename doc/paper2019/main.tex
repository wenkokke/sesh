\documentclass[sigplan,screen,review]{acmart}
\title{Rusty Variation}
\subtitle{Session Types with Failure for Rust}
\author{Wen Kokke}
\orcid{0000-0002-1662-0381}
\affiliation{
  \department{Laboratory for Foundations of Computer Science}
  \institution{University of Edinburgh}
  \streetaddress{10 Crichton Street}
  \city{Edinburgh}
  \state{Scotland}
  \postcode{EH8 9AB}
  \country{United Kingdom}
}
\email{wen.kokke@ed.ac.uk}
\setcopyright{none}
\settopmatter{printccs=false, printacmref=false, printfolios=false}

\usepackage{textcomp}
\usepackage{textgreek}
\usepackage{doi}
\usepackage[all]{foreign}

% compact enums
\usepackage{enumitem}
\setlist[enumerate]{nosep}

% red squiggly
\usepackage{ulem}
\normalem%
\makeatletter
\def\uwave{\bgroup \markoverwith{\lower3.5\p@\hbox{\sixly \textcolor{red}{\char58}}}\ULon}
\font\sixly=lasy6 % does not re-load if already loaded, so no memory problem.
\makeatother

% load listings-rust
\usepackage{inconsolata}
\usepackage{listings}
\usepackage{listings-rust}
\lstset{
  language=Rust,
  style=colouredRust,
  basicstyle=\ttfamily\small,
  columns=fullflexible,
  xleftmargin=\parindent,
  moredelim=[is][\uwave]{~}{~}
}

% load cleverref
\usepackage{cleveref}
\usepackage[nomarkers,figuresonly]{endfloat}

\begin{document}
\maketitle

\section{Exceptional GV}

\begin{figure*}
  \begin{mdframed}\begin{highlight}
    \centering
    \(
    \begin{array}{llrl}
      \text{Variables}
      &\tm{x}, \tm{y}, \tm{z}
      \\
      \text{Types}
      &\ty{A}, \ty{B}, \ty{C}
      &::=& \gvTyUnit
            \sep \gvTyFun{A}{B}
            \sep \gvTySum{A}{B}
            \sep \gvTyPair{A}{B}
            \sep \ty{S}
      \\
      \text{Session types}
      &\ty{S}, \ty{T}
      &::=& \gvTySend{A}{S}
            \sep \gvTyRecv{A}{S}
            \sep \gvTyEnd
      \\
      \text{Environments}
      &\ty{\Gamma}, \ty{\Delta}
      &::= & \emptyenv
             \sep \ty{\Gamma}, \gvVar{x} : \ty{S}
      \\
      \text{Terms}
      &\tm{L}, \tm{M}, \tm{N}
      &::=& \gvVar{x}
            \sep \gvAbs[A]{x}{M}
            \sep \gvApp{M}{N}
      \\
      &
      &\mid& \gvUnit
             \sep \gvLetUnit{M}{N}
      \\
      &
      &\mid& \gvPair{M}{N}
             \sep \gvLetPair{x}{y}{M}{N}
      \\
      &
      &\mid& \gvInl{M}
             \sep \gvInr{M}
             \sep \gvCaseSum{L}{x}{M}{y}{N}
      \\
      &
      &\mid& \gvRaise
             \sep \gvTry{L}{x}{M}{N}
      \\
      &
      &\mid&\gvFork{M}
             \sep \gvSend{M}{N}
             \sep \gvRecv{M}
             \sep \gvClose{M}
             \sep \gvCancel{M}
    \end{array}
    \)
  \end{highlight}\end{mdframed}
  \caption{Exceptional GV, static syntax.}
  \label{fig:egv-static-syntax}
\end{figure*}

%%% Local Variables:
%%% TeX-master: "main"
%%% End:

\begin{figure*}
  \begin{mdframed}
    \setstretch{2.5}

    \header{Term Typing}{\seq{\ty{\Gamma}}{M}{A}}
    \begin{center} 
      \begin{prooftree*}
        \AXC{}
        \RightLabel{\textsc{T-Var}}
        \UIC{$\seq{\tmty{x}{A}}{\gvVar{x}}{A}$}
      \end{prooftree*}
      \begin{prooftree*}
        \AXC{$\seq{\ty{\Gamma}\st{,}\;\tmty{x}{A}}{M}{B}$}
        \RightLabel{\textsc{T-Abs}}
        \UIC{$\seq{\ty{\Gamma}}{\gvAbs[A]{x}{M}}{\gvTyFun{A}{B}}$}
      \end{prooftree*}
      \begin{prooftree*}
        \AXC{$\seq{\ty{\Gamma}}{M}{\gvTyFun{A}{B}}$}
        \AXC{$\seq{\ty{\Delta}}{N}{A}$}
        \RightLabel{\textsc{T-App}}
        \BIC{$\seq{\ty{\Gamma}\st{,}\;\ty{\Delta}}{\gvApp{M}{N}}{B}$}
      \end{prooftree*}
      
      \begin{prooftree*}
        \AXC{}
        \RightLabel{\textsc{T-Unit}}
        \UIC{$\seq{\emptyenv}{\gvUnit}{\gvTyUnit}$}
      \end{prooftree*}
      \begin{prooftree*}
        \AXC{$\seq{\ty{\Gamma}}{M}{\gvTyUnit}$}
        \AXC{$\seq{\ty{\Delta}}{N}{A}$}
        \RightLabel{\textsc{T-LetUnit}}
        \BIC{$\seq{\ty{\Gamma}\st{,}\;\ty{\Delta}}{\gvLetUnit{M}{N}}{A}$}
      \end{prooftree*}
      
      \begin{prooftree*}
        \AXC{$\seq{\ty{\Gamma}}{M}{A}$}
        \AXC{$\seq{\ty{\Delta}}{N}{B}$}
        \RightLabel{\textsc{T-Pair}}
        \BIC{$\seq{\ty{\Gamma}\st{,}\;\ty{\Delta}}{\gvPair{M}{N}}{\gvTyPair{A}{B}}$}
      \end{prooftree*}
      \begin{prooftree*}
        \AXC{$\seq{\ty{\Gamma}}{M}{\gvTyPair{A}{B}}$}
        \AXC{$\seq{\ty{\Delta}\st{,}\;\tmty{x}{A}\st{,}\;\tmty{y}{B}}{N}{C}$}
        \RightLabel{\textsc{T-LetPair}}
        \BIC{$\seq{\ty{\Gamma}\st{,}\;\ty{\Delta}}{\gvLetPair{x}{y}{M}{N}}{C}$}
      \end{prooftree*}
      
      \begin{prooftree*}
        \AXC{$\seq{\ty{\Gamma}}{M}{A}$}
        \RightLabel{\textsc{T-Left}}
        \UIC{$\seq{\ty{\Gamma}}{\gvInl{M}}{\gvTySum{A}{B}}$}
      \end{prooftree*}
      \begin{prooftree*}
        \AXC{$\seq{\ty{\Gamma}}{M}{B}$}
        \RightLabel{\textsc{T-Right}}
        \UIC{$\seq{\ty{\Gamma}}{\gvInr{M}}{\gvTySum{A}{B}}$}
      \end{prooftree*}
      
      \begin{prooftree*}
        \AXC{$\seq{\ty{\Gamma}}{L}{\gvTySum{A}{B}}$}
        \AXC{$\seq{\ty{\Delta}\st{,}\;\tmty{x}{A}}{M}{C}$}
        \AXC{$\seq{\ty{\Delta}\st{,}\;\tmty{y}{B}}{N}{C}$}
        \RightLabel{\textsc{T-CaseSum}}
        \TIC{$\seq{\ty{\Gamma}\st{,}\;\ty{\Delta}}{\gvCaseSum{L}{x}{M}{y}{N}}{C}$}
      \end{prooftree*}
      
      \begin{prooftree*}
        \AXC{$\seq{\ty{\Gamma}}{M}{A}$}
        \RightLabel{\textsc{T-Cancel}}
        \UIC{$\seq{\ty{\Gamma}}{\gvCancel{M}}{\gvUnit}$}
      \end{prooftree*}
      \begin{prooftree*}
        \AXC{$\seq{\ty{\Gamma}}{L}{A}$}
        \AXC{$\seq{\ty{\Delta}\st{,}\;\tmty{x}{A}}{M}{B}$}
        \AXC{$\seq{\ty{\Gamma}}{N}{B}$}
        \RightLabel{\textsc{T-Try}}
        \TIC{$\seq{\ty{\Gamma}\st{,}\;\ty{\Delta}}{\gvTry{L}{x}{M}{N}}{B}$}
      \end{prooftree*}
      \begin{prooftree*}
        \AXC{}
        \RightLabel{\textsc{T-Raise}}
        \UIC{$\seq{\emptyenv}{\gvRaise}{A}$}
      \end{prooftree*}
      
      \begin{prooftree*}
        \AXC{$\seq{\ty{\Gamma}}{M}{\gvTyFun{S}{\gvTyUnit}}$}
        \RightLabel{\textsc{T-Fork}}
        \UIC{$\seq{\ty{\Gamma}}{\gvFork{M}}{\gvDual{S}}$}
      \end{prooftree*}    
      \begin{prooftree*}
        \AXC{$\seq{\ty{\Gamma}}{M}{\gvTyEnd}$}
        \RightLabel{\textsc{T-Close}}
        \UIC{$\seq{\ty{\Gamma}}{\gvClose{M}}{\gvUnit}$}
      \end{prooftree*}
      
      \begin{prooftree*}
        \AXC{$\seq{\ty{\Gamma}}{M}{A}$}
        \AXC{$\seq{\ty{\Delta}}{N}{\gvTySend{A}{S}}$}
        \RightLabel{\textsc{T-Send}}
        \BIC{$\seq{\ty{\Gamma}\st{,}\;\ty{\Delta}}{\gvSend{M}{N}}{S}$}
      \end{prooftree*}
      \begin{prooftree*}
        \AXC{$\seq{\ty{\Gamma}}{M}{\gvTyRecv{A}{S}}$}
        \RightLabel{\textsc{T-Recv}}
        \UIC{$\seq{\ty{\Gamma}}{\gvRecv{M}}{\gvPair{A}{S}}$}
      \end{prooftree*}
    \end{center}
      
    \header{Duality}{\gvDual{S}}
    \begin{center}
      \(
      \gvDual{\gvTySend{A}{S}}\;=\;\gvTyRecv{A}{\gvDual{S}}
      \quad
      \gvDual{\gvTyRecv{A}{S}}\;=\;\gvTySend{A}{\gvDual{S}}
      \quad
      \gvDual{\gvTyEnd}\;=\;\gvTyEnd
      \)
    \end{center}
  \end{mdframed}
\caption{Exceptional GV, static typing.}
\label{fig:egv-static-typing}
\end{figure*}

%%% Local Variables:
%%% TeX-master: "main"
%%% End:
\begin{figure*}
  \begin{mdframed}\begin{highlight}
    \centering
    \(
    \begin{array}{llrl}
      \text{Channels}
      &\tm{a}, \tm{b}, \tm{c}
      \\
      \text{Runtime types}
      &\ty{R}
      &::= & \ty{S}
             \sep \gvTyLock{S}
      \\
      \text{Environments}
      &\ty{\Gamma}, \ty{\Delta}
      &::= & \dots
             \sep \ty{\Gamma}, \gvVar{a} : \ty{S}
      \\
      \text{Runtime environments}
      &\ty{\Psi, \Phi}
      &::= & \emptyenv
             \sep \ty{\Psi}, \gvVar{a} : \ty{R}
      \\
      \text{Terms}
      &\tm{L}, \tm{M}, \tm{N}
      &::= & \dots
             \sep \tm{a}
      \\
      \text{Values}
      &\tm{U}, \tm{V}, \tm{W}
      &::= & \tm{a}
             \sep \gvAbs[A]{x}{M}
             \sep \gvUnit
             \sep \gvPair{V}{W}
             \sep \gvInl{V}
             \sep \gvInr{V}
      \\
      \text{Flags}
      &\tm{\phi}
      &::= & \gvFlagMain \sep \gvFlagChild
      \\
      \text{Configurations}
      &\tm{\gvConf{C}}, \tm{\gvConf{D}}, \tm{\gvConf{E}}
      &::= & \tm{\phi}\tm{M}
             \sep \gvRes{a}{\gvConf{C}}
             \sep \gvPPar{\gvConf{C}}{\gvConf{D}}
             \sep \gvHalt
             \sep \gvZap{a}
             \sep \gvBuf{a}{\gvVec{V}}{b}{\gvVec{W}}
      \\
      \text{Evaluation contexts}
      &\tm{E}
      &::= & \gvHole
             \sep \gvApp{E}{M}
             \sep \gvApp{V}{E}
             \sep \gvLetUnit{E}{M}
             \sep \gvPair{E}{M}
             \sep \gvPair{V}{E}
             \sep \gvLetPair{x}{y}{E}{M}
      \\
      &
      &\mid& \gvInl{E}
             \sep \gvInr{E}
             \sep \gvCaseSum{E}{x}{M}{y}{N}
      \\
      &
      &\mid& \gvFork{E}
             \sep \gvSend{E}{M}
             \sep \gvSend{V}{E}
             \sep \gvRecv{E}
             \sep \gvClose{E}
             \sep \gvCancel{E}
      \\
      &
      &\mid& \gvTry{E}{x}{M}{N}
      \\
      \text{Pure contexts}
      &\tm{P}
      &::= & \gvHole
             \sep \gvApp{P}{M}
             \sep \gvApp{V}{P}
             \sep \gvLetUnit{P}{M}
             \sep \gvPair{P}{M}
             \sep \gvPair{V}{P}
             \sep \gvLetPair{x}{y}{P}{M}
      \\
      &
      &\mid& \gvInl{P}
             \sep \gvInr{P}
             \sep \gvCaseSum{P}{x}{M}{y}{N}
      \\
      &
      &\mid& \gvFork{P}
             \sep \gvSend{P}{M}
             \sep \gvSend{V}{P}
             \sep \gvRecv{P}
             \sep \gvClose{P}
             \sep \gvCancel{P}
      \\
      \text{Thread contexts}
      &\tm{\gvConf{F}}
      &::= & \tm{\phi}\tm{E}
      \\
      \text{Configuration contexts}
      &\tm{\gvConf{G}}
      &::= & \gvHole
             \sep \gvRes{a}{\gvConf{G}}
             \sep \gvPPar{\gvConf{G}}{\gvConf{C}}
    \end{array}
    \)
  \end{highlight}\end{mdframed}
  \caption{Exceptional GV, runtime syntax.}
  \label{fig:egv-runtime-syntax}
\end{figure*}

%%% Local Variables:
%%% TeX-master: "main"
%%% End:

\begin{figure*}
  \begin{mdframed}\begin{highlight}
    \begin{minipage}[t]{0.25\textwidth}
      \header{Term typing}{\gvSeq{\ty{\Gamma}}{M}{A}}
      \\
      \begin{center}
        \begin{prooftree*}
          \AXC{$\vphantom{\Gamma}$}
          \RightLabel{\rlabel{\textsc{T-Name}}{rule:egv-ty-name}}
          \UIC{$\gvSeq{\tmty{a}{S}}{a}{S}$}
        \end{prooftree*}
      \end{center}
    \end{minipage}%
    \hspace*{0.05\textwidth}%
    \begin{minipage}[t]{0.25\textwidth}
      \header{Session slicing}{\gvSlice{S}{\gvTyVec{A}}}
      \begin{gather*}
        \gvSlice{S}{\gvVecEmp}
        = \ty{S}
        \\
        \gvSlice{\gvTySend{A}{S}}{\ty{A}\gvVecAdd\gvVec{\ty{A}}}
        = \gvSlice{S}{\gvTyVec{A}}
      \end{gather*}
    \end{minipage}%
    \hspace*{0.05\textwidth}%
    \begin{minipage}[t]{0.4\textwidth}
      \header{Queue typing}{\gvSeq{\ty{\Gamma}}{\gvVec{V}}{\gvTyVec{A}}}
      \\
      \begin{center}
        \begin{prooftree*}
          \AXC{$\vphantom{\Gamma}$}
          \UIC{$\gvSeq{\emptyenv}{\gvVecEmp}{\gvVecEmp}$}
        \end{prooftree*}%
        \begin{prooftree*}
          \AXC{$\gvSeq{\ty{\Gamma}}{V}{A}$}
          \AXC{$\gvSeq{\ty{\Delta}}{\gvVec{V}}{\gvTyVec{A}}$}
          \BIC{$\gvSeq{\ty{\Gamma}\st{,}\;\ty{\Delta}}{\tm{V}\gvVecAdd\gvVec{V}}{\ty{A}\gvVecAdd\gvTyVec{A}}$}
        \end{prooftree*}
      \end{center}
    \end{minipage}%
    \vspace{1\baselineskip}
    \header{Configuration typing}{\gvCSeq{\ty{\Gamma}}{\ty{\Phi}}{\phi}{\gvConf{C}}}
    \begin{center}
      \begin{prooftree*}
        \AXC{$\gvCSeq{\ty{\Gamma}}{\ty{\Phi}\st{,}\;\tmty{a}{\gvTyLock{S}}}{\phi}{\gvConf{C}}$}
        \RightLabel{\rlabel{\textsc{T-New}}{rule:egv-ty-new}}
        \UIC{$\gvCSeq{\ty{\Gamma}}{\ty{\Phi}}{\phi}{\gvRes{a}{\gvConf{C}}}$}
      \end{prooftree*}
      \begin{prooftree*}
        \AXC{$\gvCSeq{\ty{\Gamma}}{\ty{\Phi}}{\phi}{\gvConf{C}}$}
        \AXC{$\gvCSeq{\ty{\Delta}}{\ty{\Psi}}{\psi}{\gvConf{D}}$}
        \RightLabel{\rlabel{\textsc{T-Par}}{rule:egv-ty-par}}
        \BIC{$\gvCSeq%
          {\ty{\Gamma}\st{,}\;\ty{\Delta}}%
          {\ty{\Phi}\st{,}\;\ty{\Psi}}%
          {\phi+\psi}%
          {\gvPar{\gvConf{C}}{\gvConf{D}}}$}
      \end{prooftree*}
      \setstretch{2.5}

      \begin{prooftree*}
        \AXC{$\gvCSeq{\ty{\Gamma}}{\ty{\Phi}\st{,}\;\tmty{a}{S}}{\phi}{\gvConf{C}}$}
        \AXC{$\gvCSeq{\ty{\Delta}}{\ty{\Psi}\st{,}\;\tmty{a}{\gvDual{S}}}{\psi}{\gvConf{D}}$}
        \RightLabel{\rlabel{\textsc{T-Syn}$_1$}{rule:egv-ty-syn1}}
        \BIC{$\gvCSeq%
          {\ty{\Gamma}\st{,}\;\ty{\Delta}}%
          {\ty{\Phi}\st{,}\;\ty{\Psi}\st{,}\;\tmty{a}{\gvTyLock{S}}}%
          {\phi+\psi}%
          {\gvPar{\gvConf{C}}{\gvConf{D}}}$}
      \end{prooftree*}
      \begin{prooftree*}
        \AXC{$\gvCSeq{\ty{\Gamma}}{\ty{\Phi}\st{,}\;\tmty{a}{\gvDual{S}}}{\phi}{\gvConf{C}}$}
        \AXC{$\gvCSeq{\ty{\Delta}}{\ty{\Psi}\st{,}\;\tmty{a}{S}}{\psi}{\gvConf{D}}$}
        \RightLabel{\rlabel{\textsc{T-Syn}$_2$}{rule:egv-ty-syn2}}
        \BIC{$\gvCSeq%
          {\ty{\Gamma}\st{,}\;\ty{\Delta}}%
          {\ty{\Phi}\st{,}\;\ty{\Psi}\st{,}\;\tmty{a}{\gvTyLock{S}}}%
          {\phi+\psi}%
          {\gvPar{\gvConf{C}}{\gvConf{D}}}$}
      \end{prooftree*}

      \begin{prooftree*}
        \AXC{$\gvSeq{\ty{\Gamma}}{M}{A}$}
        \RightLabel{\rlabel{\textsc{T-Main}}{rule:egv-ty-main}}
        \UIC{$\gvCSeq{\ty{\Gamma}}{\emptyenv}{\gvFlagMain}{\gvMain{M}}$}
      \end{prooftree*}
      \begin{prooftree*}
        \AXC{$\gvSeq{\ty{\Gamma}}{M}{\gvTyUnit}$}
        \RightLabel{\rlabel{\textsc{T-Child}}{rule:egv-ty-child}}
        \UIC{$\gvCSeq{\ty{\Gamma}}{\emptyenv}{\gvFlagChild}{\gvChild{M}}$}
      \end{prooftree*}
      \begin{prooftree*}
        \AXC{$\vphantom{\Gamma}$}
        \RightLabel{\rlabel{\textsc{T-Halt}}{rule:egv-ty-halt}}
        \UIC{$\gvCSeq{\emptyenv}{\emptyenv}{\gvFlagMain}{\gvHalt}$}
      \end{prooftree*}
      \begin{prooftree*}
        \AXC{$\vphantom{\Gamma}$}
        \RightLabel{\rlabel{\textsc{T-Zap}}{rule:egv-ty-zap}}
        \UIC{$\gvCSeq{\tmty{a}{S}}{\emptyenv}{\gvFlagChild}{\gvZap{a}}$}
      \end{prooftree*}

      \begin{prooftree*}
        \AXC{$\gvSeq{\ty{\Gamma}\st{,}\;\tmty{a}{S}}{\gvVec{V}}{\gvTyVec{A}}$}
        \AXC{$\gvSeq{\ty{\Delta}\st{,}\;\tmty{a}{T}}{\gvVec{W}}{\gvTyVec{B}}$}
        \AXC{$\gvSlice{S}{\gvTyVec{A}}=\gvDual{\gvSlice{T}{\gvTyVec{B}}}$}
        \RightLabel{\rlabel{\textsc{T-Buf}}{rule:egv-ty-buf}}
        \TIC{$\gvCSeq%
          {\ty{\Gamma}\st{,}\;\ty{\Delta}}%
          {\tmty{a}{S}\st{,}\;\tmty{b}{\gvDual{S}}}%
          {\gvFlagChild}%
          {\gvBuf{a}{\gvVec{V}}{b}{\gvVec{W}}}$}
      \end{prooftree*}
    \end{center}
    \vspace*{1\baselineskip}

    \begin{minipage}[t]{0.4\textwidth}
      \header{Flag combination}{\phi+\psi}
      \[
        \begin{array}{ll}
          \gvFlagMain  + \gvFlagChild = \gvFlagMain
          &
            \gvFlagChild + \gvFlagMain  = \gvFlagMain
          \\
          \gvFlagChild + \gvFlagChild = \gvFlagChild
          &
            \gvFlagMain  + \gvFlagMain \; \text{undefined}
        \end{array}
      \]
      \header{Session type reduction}{\gvRed{\ty{S}}{\ty{T}}}
      \[
        \gvRed{\ty{\gvTySend{A}{S}}}{\ty{S}}
        \qquad
        \gvRed{\ty{\gvTyRecv{A}{S}}}{\ty{S}}
      \]
    \end{minipage}%
    \hspace*{0.05\textwidth}%
    \begin{minipage}[t]{0.55\textwidth}
      \header{Environment reduction}{\gvRed{\ty{\Gamma}\st{;}\;\ty{\Phi}}{\ty{\Delta}\st{;}\;\ty{\Psi}}}
      \begin{center}
        \begin{prooftree*}
          \AXC{$\gvRed{\ty{S}}{\ty{T}}$}
          \UIC{$\gvRed
            {\ty{\Gamma}\st{,}\;\tmty{a}{S}\st{;}\;\ty{\Phi}}
            {\ty{\Gamma}\st{,}\;\tmty{a}{T}\st{;}\;\ty{\Phi}}$}
        \end{prooftree*}

        \setstretch{2.5}
        \begin{prooftree*}
          \AXC{$\gvRed{\ty{S}}{\ty{T}}$}
          \UIC{$\gvRed
            {\ty{\Gamma}\st{;}\;\ty{\Phi}\st{,}\;\tmty{a}{S}}
            {\ty{\Gamma}\st{;}\;\ty{\Phi}\st{,}\;\tmty{a}{T}}$}
        \end{prooftree*}
        \begin{prooftree*}
          \AXC{$\gvRed{\ty{S}}{\ty{T}}$}
          \UIC{$\gvRed
            {\ty{\Gamma}\st{;}\;\ty{\Phi}\st{,}\;\tmty{a}{\gvTyLock{S}}}
            {\ty{\Gamma}\st{;}\;\ty{\Phi}\st{,}\;\tmty{a}{\gvTyLock{T}}}$}
        \end{prooftree*}
      \end{center}
    \end{minipage}%
  \end{highlight}\end{mdframed}
  \caption{Exceptional GV, runtime typing.}
  \label{fig:egv-runtime-typing}
\end{figure*}

%%% Local Variables:
%%% TeX-master: "main"
%%% End:

\begin{figure*}
    \begin{mdframed}
    {Term reduction}
    \[\!\!%
      \setlength{\arraycolsep}{4pt}%
      \begin{array}{llcl}
        \textsc{E-Lam}
        & {\gvApp{\gvAbs[A]{x}{M}}{V}}
        & \gvRedArrPure
        & {\gvSub{M}{x}{V}}
        \\
        \textsc{E-Unit}
        & {\gvLetUnit{\gvUnit}{M}}
        & \gvRedArrPure
        & {\tm{M}}
        \\
        \textsc{E-Pair}
        & {\gvLetPair{x}{y}{\gvPair{V}{W}}{M}}
        & \gvRedArrPure
        & {\gvSub{M}{\tm{x}\st{,}\;\tm{y}}{\tm{V}\st{,}\;\tm{W}}}
        \\
        \textsc{E-Inl}
        & {\gvCaseSum{\gvInl{V}}{x}{M}{y}{N}}
        & \gvRedArrPure
        & {\gvSub{M}{x}{V}}
        \\
        \textsc{E-Inr}
        & {\gvCaseSum{\gvInr{V}}{x}{M}{y}{N}}
        & \gvRedArrPure
        & {\gvSub{N}{y}{V}}
        \\
        \textsc{E-Val}
        & {\gvTry{V}{x}{M}{N}}
        & \gvRedArrPure
        & {\gvSub{M}{x}{V}}
        \\
        \textsc{E-Lift}
        & {\gvPlug{E}{M}}
        & \gvRedArrPure
        & {\gvPlug{E}{M'}}\st{,}
          \quad\text{if}\;\gvRedPure{M}{M'}
      \end{array}
    \]
    {Configuration equivalence}
    \[
      \gvPar{\gvConf{C}}{\gvPPar{\gvConf{D}}{\gvConf{E}}}
      \equiv
      \gvPar{\gvPPar{\gvConf{C}}{\gvConf{D}}}{\gvConf{E}}
      \qquad
      \gvPar{\gvConf{C}}{\gvConf{D}}
      \equiv
      \gvPar{\gvConf{D}}{\gvConf{C}}
      \qquad
      \gvRes{a}{\gvRes{b}{\gvConf{C}}}
      \equiv
      \gvRes{b}{\gvRes{a}{\gvConf{C}}}
    \]
    \[
      \gvBuf{a}{\gvVec{V}}{b}{\gvVec{W}}
      \equiv
      \gvBuf{b}{\gvVec{W}}{a}{\gvVec{V}}
      \qquad
      \gvPar{\gvConf{C}}{\gvRes{a}{\gvConf{D}}}
      \equiv
      \gvRes{a}{\gvPPar{\gvConf{C}}{\gvConf{D}}},
      \quad\text{if}\;\gvVar{a}\notin\text{fv}{(\gvConf{C})}
    \]
    {Configuration reduction}
    \[\!\!%
      \setlength{\arraycolsep}{4pt}%
      \begin{array}{llcl}
        \textsc{E-Fork}
        & \multicolumn{3}{l}{%
          {\gvPlug{\gvConf{F}}{\gvFork{\gvAbs[A]{x}{M}}}}
          \gvRedArr
          {\gvRes{a}{\gvRes{b}{\gvPPar{\gvPlug{\gvConf{F}}{a}}{%
          \gvPar{\gvChild{\gvSub{M}{b}{x}}}{\gvBuf{a}{\gvVecEmp}{b}{\gvVecEmp}}}}}},
          \quad\text{where \tm{a}, \tm{b} are fresh}}
        \\
        \textsc{E-Send}
        & {\gvPar{\gvPlug{\gvConf{F}}{\gvSend{U}{a}}}{%
          \gvBuf{a}{\gvVec{V}}{b}{\gvVec{W}}}}
        & \gvRedArr
        & {\gvPar{\gvPlug{\gvConf{F}}{a}}{%
          \gvBuf{a}{\gvVec{V}}{b}{\gvVec{W}\gvVecAdd\gvVar{U}}}}
        \\
        \textsc{E-Recv}
        & {\gvPar{\gvPlug{\gvConf{F}}{\gvRecv{U}}}{%
          \gvBuf{a}{\gvVar{U}\gvVecAdd\gvVec{V}}{b}{\gvVec{W}}}}
        & \gvRedArr
        & {\gvPar{\gvPlug{\gvConf{F}}{\gvPair{U}{a}}}{%
          \gvBuf{a}{\gvVec{V}}{b}{\gvVec{W}}}}
        \\
        \textsc{E-Close}
        & {\gvPar{\gvPlug{\gvConf{F}}{\gvClose{a}}}{%
          \gvPar{\gvPlug{\gvConf{F'}}{\gvClose{b}}}{%
          \gvBuf{a}{\gvVecEmp}{b}{\gvVecEmp}}}}
        & \gvRedArr
        & {\gvPar{\gvPlug{\gvConf{F}}{\gvUnit}}{%
          \gvPlug{\gvConf{F'}}{\gvUnit}}}
        \\
        \textsc{E-Cancel}
        & {\gvPlug{\gvConf{F}}{\gvCancel{a}}}
        & \gvRedArr
        & {\gvPar{\gvPlug{\gvConf{F}}{\gvUnit}}{\gvZap{a}}}
        \\
        \textsc{E-Zap}
        & {\gvPar{\gvZap{a}}{\gvBuf{a}{\gvVar{U}\gvVecAdd\gvVec{V}}{b}{\gvVec{W}}}}
        & \gvRedArr
        & {\gvPar{\gvZap{a}}{\gvPar{\gvZap{U}}{\gvBuf{a}{\gvVec{V}}{b}{\gvVec{W}}}}}
        \\
        \textsc{E-CloseZap}
        & {\gvPar{\gvPlug{\gvConf{F}}{\gvClose{a}}}{\gvPar{\gvZap{b}}{%
          \gvBuf{a}{\gvVecEmp}{b}{\gvVecEmp}}}}
        & \gvRedArr
        & {\gvPar{\gvPlug{\gvConf{F}}{\gvRaise}}{\gvPar{\gvZap{a}}{%
          \gvPar{\gvZap{b}}{\gvBuf{a}{\gvVecEmp}{b}{\gvVecEmp}}}}}
        \\
        \textsc{E-RecvZap}
        & {\gvPar{\gvPlug{\gvConf{F}}{\gvRecv{a}}}{\gvPar{\gvZap{b}}{%
          \gvBuf{a}{\gvVecEmp}{b}{\gvVec{W}}}}}
        & \gvRedArr
        & {\gvPar{\gvPlug{\gvConf{F}}{\gvRaise}}{\gvPar{\gvZap{a}}{%
          \gvPar{\gvZap{b}}{\gvBuf{a}{\gvVecEmp}{b}{\gvVec{W}}}}}}
        \\
        \textsc{E-Raise}
        & {\gvPlug{\gvConf{F}}{\gvTry{\gvPlug{P}{\gvRaise}}{x}{M}{N}}}
        & \gvRedArr
        & {\gvPar{\gvPlug{\gvConf{F}}{N}}{\gvZap{P}}}
        \\
        \textsc{E-RaiseChild}
        & {\gvChild{\gvPlug{P}{\gvRaise}}}
        & \gvRedArr
        & {\gvZap{P}}
        \\
        \textsc{E-RaiseMain}
        & {\gvMain{\gvPlug{P}{\gvRaise}}}
        & \gvRedArr
        & {\gvPar{\gvHalt}{\gvZap{P}}}
        \\
        \textsc{E-HaltChild}
        & {\gvPar{\gvConf{C}}{\gvChild{\gvUnit}}}
        & \gvRedArr
        & {\tm{\gvConf{C}}}
        \\
        \textsc{E-GC}
        & {\gvPar{\gvRes{a}{\gvRes{b}{\gvPPar{\gvZap{a}}{\gvPar{\gvZap{b}}{%
          \gvBuf{a}{\gvVecEmp}{b}{\gvVecEmp}}}}}}{\gvConf{C}}}
        & \gvRedArr
        & {\tm{\gvConf{C}}}
        \\
        \textsc{E-LiftC}
        & {\gvPlug{\gvConf{G}}{\gvConf{C}}}
        & \gvRedArr
        & {\gvPlug{\gvConf{G}}{\gvConf{D}}},
          \quad{\tm{\gvConf{C}}\gvRedArr\tm{\gvConf{D}}}
        \\
        \textsc{E-LiftM}
        & {\gvThread{\phi}{M}}
        & \gvRedArr
        & {\gvThread{\phi}{M'}},
          \quad{\gvThread{\phi}{M}\gvRedArrPure\gvThread{\phi}{M'}}
      \end{array} 
    \]
  \end{mdframed}
  \caption{Exceptional GV, reduction semantics.}
  \label{fig:egv-reduction}
\end{figure*}

\begin{figure*}
  \begin{mdframed}\begin{highlight}
    Definition of \textbf{do}-notation for \affineAGV:%
    \[
      \begin{array}{lcl}
        \gvDo{x}{M}{N} &::=& \gvDoDef{x}{M}{N}
        \\
        \gvDo{\gvUnit}{M}{N} &::=& \gvDoDef{x}{M}{\gvLetUnit{x}{N}}
        \\
        \gvDo{\gvPair{x}{y}}{M}{N} &::=& \gvDoDef{z}{M}{\gvLetPair{x}{y}{z}{N}}
      \end{array}
    \] 
    Definition of $\ftm{\cdot}$ on types and session types:%
    \begin{center}
      \begin{minipage}[t]{0.5\linewidth}
        \[
          \begin{array}{lcl}
            \ftm{\gvTyUnit}         & = & \gvTyUnit
            \\
            \ftm{{\gvTyFun{A}{B}}}  & = & \gvTyFun{\ftm{A}}{\gvTyOpt{\ftm{B}}}
            \\
            \ftm{{\gvTySum{A}{B}}}  & = & \gvTySum{\ftm{A}}{\ftm{B}}
            \\
            \ftm{{\gvTyPair{A}{B}}} & = & \gvTyPair{\ftm{A}}{\ftm{B}}
          \end{array}
        \]
      \end{minipage}%
      \begin{minipage}[t]{0.5\linewidth}
        \[
          \begin{array}{lcl}
            \ftm{{\gvTySend{A}{S}}} & = & \gvTySend{\ftm{A}}{\ftm{S}}
            \\
            \ftm{{\gvTyRecv{A}{S}}} & = & \gvTyRecv{\ftm{A}}{\ftm{S}}
            \\
            \ftm{\gvTyEnd}          & = & \gvTyEnd
          \end{array}
        \]
      \end{minipage}
    \end{center}
    Definition of $\atm{}$ on terms:%
    \[
      \begin{array}{lcl}
        \atm{\tm{x}}
        & = & \gvSome{\tm{x}}
        \\
        \atm{\tm{a}}
        & = & \gvSome{\tm{a}}
        \\
        \atm{\parens{\gvAbs[A]{x}{M}}}
        & = & \gvSome{\parens{\gvAbs[\ftm{A}]{x}{\atm{M}}}}
        \\
        \atm{\parens{\gvApp{M}{N}}}
        & = & \gvDo{f}{\atm{M}}{{\gvDo{x}{\atm{N}}{\gvApp{f}{x}}}}
        \\
        \atm{\gvUnit}
        & = & \gvSome{\gvUnit}
        \\
        \atm{\parens{\gvLetUnit{M}{N}}}
        & = & \gvDo{\gvUnit}{\atm{M}}{{\atm{N}}}
        \\
        \atm{\gvPair{M}{N}}
        & = & \gvDo{x}{\atm{M}}{{\gvDo{y}{\atm{N}}{{\gvSome{\gvPair{x}{y}}}}}}
        \\
        \atm{\parens{\gvLetPair{x}{y}{M}{N}}}
        & = & \gvDo{\gvPair{x}{y}}{\atm{M}}{\atm{N}}
        \\
        \atm{\parens{\gvInl{M}}}
        & = & \gvDo{x}{\atm{M}}{\gvSome{\parens{\gvInl{x}}}}
        \\
        \atm{\parens{\gvInr{M}}}
        & = & \gvDo{x}{\atm{M}}{\gvSome{\parens{\gvInr{x}}}}
        \\
        \atm{\parens{\gvCaseSum{L}{x}{M}{y}{N}}}
        & = & \gvDo{z}{L}{{\gvCaseSum{z}{x}{\atm{M}}{y}{\atm{N}}}}
        \\
        \atm{\parens{\gvCancel{M}}}
        & = & \gvDo{\gvUnit}{\atm{M}}{\gvSome{\gvUnit}}
        \\
        \atm{\parens{\gvFork{M}}}
        & = & \gvSome{\parens{\gvFork{\atm{M}}}}
        \\
        \atm{\parens{\gvSend{M}{N}}}
        & = & \gvDo{x}{\atm{M}}{{\gvDo{y}{\atm{N}}{\gvSome{\parens{\gvSend{x}{y}}}}}}
        \\
        \atm{\parens{\gvRecv{M}}}
        & = & \gvDo{x}{\atm{M}}{{\gvRecv{x}}}
        \\
        \atm{\parens{\gvClose{M}}}
        & = & \gvSome{\parens{\gvClose{M}}}
        \\[1ex]
        \multicolumn{3}{c}{%
        \text{(where all variables introduced on the right-hand side are fresh)}}
      \end{array}
    \]
  \end{highlight}\end{mdframed}
  \caption[\affineEGV to \affineAGV]{Translation from \affineEGV to \affineAGV.}
  \label{fig:affine-egv-to-agv}
\end{figure*}


\section{Rusty Variation}

\begin{figure*}
  \begin{mdframed}
    \[
    \begin{array}{llrl}
      \text{Variables}
      &\tm{x}, \tm{y}, \tm{z}
      \\
      \text{Types}
      &\ty{A}, \ty{B}, \ty{C}
      &::=& \rvTyUnit
            \sep \rvTyFun{A}{B}
            \sep \rvTySum{A}{B}
            \sep \rvTyPair{A}{B}
            \sep \texttt{\ty{S}}
      \\
      \text{Session types}
      &\ty{S}, \ty{T}
      &::=& \rvTySend{A}{S}
            \sep \rvTyRecv{A}{S}
            \sep \rvTyEnd
      \\
      \text{Terms}
      &\tm{L}, \tm{M}, \tm{N}
      &::=& \rvTm{x}
            \sep \rvAbs[A]{x}{M}
            \sep \rvApp{M}{N}
      \\
      &
      &\mid& \rvUnit
             \sep \rvLetUnit{M}{N}
     \\
      &
      &\mid& \rvPair{M}{N}
             \sep \rvLetPair{x}{y}{M}{N}
      \\
      &
      &\mid& \rvInl{M}
             \sep \rvInr{M}
             \sep \rvCaseSum{L}{x}{M}{x}{N}
      \\
      &
      &\mid& \rvFork{M}
             \sep \rvSend{M}{N}
             \sep \rvRecv{M}
    \end{array}
    \]
  \end{mdframed}
  \caption{Rusty Variation, static syntax.}
  \label{fig:rv-static-syntax}
\end{figure*}

%%% Local Variables:
%%% TeX-master: "main"
%%% End:

\begin{figure*}
  \begin{mdframed}\begin{highlight}
    {The \rvTyOpt{A} type}
    \[
      \begin{array}{lcl}
        \rvTyOpt{A} &::=& \rvTySum{A}{\rvTyUnit}
        \\
        \rvSome{x} &::=& \rvInl{x}
        \\
        \rvNone &::=& \rvInr{\rvUnit}
        \\
        \rvCaseOpt{L}{x}{M}{N} &::=& \rvCaseSum{L}{x}{M}{y}{\rvLetUnit{y}{N}}
      \end{array}
    \]
    {The \texttt{try!} macro}
    \[
      \begin{array}{lcl}
        \rvDo{x}{M}{N} &::=& \rvDoDef{x}{M}{N}
        \\
        \rvDo{\rvUnit}{M}{N} &::=& \rvDoDef{x}{M}{\rvLetUnit{x}{N}}
        \\
        \rvDo{\rvPair{x}{y}}{M}{N} &::=& \rvDoDef{z}{M}{\rvLetPair{x}{y}{z}{N}}
      \end{array}
    \]
  \end{highlight}\end{mdframed}
  \caption{Rusty Variation, syntactic sugar.}
  \label{fig:rv-syntactic-sugar}
\end{figure*}

%%% Local Variables:
%%% TeX-master: "main"
%%% End:

\begin{figure*}
  \begin{mdframed}\begin{highlight}

    \header{Term typing}{\rvSeq{\ty{\Gamma}}{\tm{M}}{A}}
    \begin{center}
      \begin{prooftree*}
        \AXC{$\tmty{x}{A}\in\ty{\Gamma}$}
        \RightLabel{\rlabel{\textsc{T-Var}}{rule:rv-ty-var}}
        \UIC{$\rvSeq{\ty{\Gamma}}{\tm{x}}{A}$}
      \end{prooftree*}
      \begin{prooftree*}
        \AXC{$\rvSeq{\ty{\Gamma}\st{,}\;\tmty{x}{A}}{M}{B}$}
        \RightLabel{\rlabel{\textsc{T-Abs}}{rule:rv-ty-abs}}
        \UIC{$\rvSeq{\ty{\Gamma}}{\rvAbs[A]{x}{M}}{\rvTyFun{A}{B}}$}
      \end{prooftree*}
      \begin{prooftree*}
        \AXC{$\rvSeq{\ty{\Gamma}}{M}{\rvTyFun{A}{B}}$}
        \AXC{$\rvSeq{\ty{\Delta}}{N}{A}$}
        \RightLabel{\rlabel{\textsc{T-App}}{rule:rv-ty-app}}
        \BIC{$\rvSeq{\ty{\Gamma}\st{,}\;\ty{\Delta}}{\rvApp{M}{N}}{B}$}
      \end{prooftree*}
      \setstretch{2.5}
      
      \begin{prooftree*}
        \AXC{}
        \RightLabel{\rlabel{\textsc{T-Unit}}{rule:rv-ty-unit}}
        \UIC{$\rvSeq{\ty{\Gamma}}{\rvUnit}{\rvTyUnit}$}
      \end{prooftree*}
      \begin{prooftree*}
        \AXC{$\rvSeq{\ty{\Gamma}}{M}{\rvTyUnit}$}
        \AXC{$\rvSeq{\ty{\Delta}}{N}{A}$}
        \RightLabel{\rlabel{\textsc{T-LetUnit}}{rule:rv-ty-letunit}}
        \BIC{$\rvSeq{\ty{\Gamma}\st{,}\;\ty{\Delta}}{\rvLetUnit{M}{N}}{A}$}
      \end{prooftree*}
      
      \begin{prooftree*}
        \AXC{$\rvSeq{\ty{\Gamma}}{M}{A}$}
        \AXC{$\rvSeq{\ty{\Delta}}{N}{B}$}
        \RightLabel{\rlabel{\textsc{T-Pair}}{rule:rv-ty-pair}}
        \BIC{$\rvSeq{\ty{\Gamma}\st{,}\;\ty{\Delta}}{\rvPair{M}{N}}{\rvTyPair{A}{B}}$}
      \end{prooftree*}
      \begin{prooftree*}
        \AXC{$\rvSeq{\ty{\Gamma}}{M}{\rvTyPair{A}{B}}$}
        \AXC{$\rvSeq{\ty{\Delta}\st{,}\;\tmty{x}{A}\st{,}\;\tmty{y}{B}}{N}{C}$}
        \RightLabel{\rlabel{\textsc{T-LetPair}}{rule:rv-ty-letpair}}
        \BIC{$\rvSeq{\ty{\Gamma}\st{,}\;\ty{\Delta}}{\rvLetPair{x}{y}{M}{N}}{C}$}
      \end{prooftree*}
      
      \begin{prooftree*}
        \AXC{$\rvSeq{\ty{\Gamma}}{M}{A}$}
        \RightLabel{\rlabel{\textsc{T-Left}}{rule:rv-ty-left}}
        \UIC{$\rvSeq{\ty{\Gamma}}{\rvInl{M}}{\rvTySum{A}{B}}$}
      \end{prooftree*}
      \begin{prooftree*}
        \AXC{$\rvSeq{\ty{\Gamma}}{M}{B}$}
        \RightLabel{\rlabel{\textsc{T-Right}}{rule:rv-ty-right}}
        \UIC{$\rvSeq{\ty{\Gamma}}{\rvInr{M}}{\rvTySum{A}{B}}$}
      \end{prooftree*}
      
      \begin{prooftree*}
        \AXC{$\rvSeq{\ty{\Gamma}}{L}{\rvTySum{A}{B}}$}
        \AXC{$\rvSeq{\ty{\Delta}\st{,}\;\tmty{x}{A}}{M}{C}$}
        \AXC{$\rvSeq{\ty{\Delta}\st{,}\;\tmty{y}{B}}{N}{C}$}
        \RightLabel{\rlabel{\textsc{T-CaseSum}}{rule:rv-ty-casesum}}
        \TIC{$\rvSeq{\ty{\Gamma}\st{,}\;\ty{\Delta}}{\rvCaseSum{L}{x}{M}{y}{N}}{C}$}
      \end{prooftree*}
      
      \begin{prooftree*}
        \AXC{$\rvSeq{\ty{\Gamma}}{M}{\rvTyFun{S}{\rvTyOpt{\rvTyUnit}}}$}
        \RightLabel{\rlabel{\textsc{T-Fork}}{rule:rv-ty-fork}}
        \UIC{$\rvSeq{\ty{\Gamma}}{\rvFork{M}}{\rvDual{S}}$}
      \end{prooftree*}
      \begin{prooftree*}
        \AXC{$\rvSeq{\ty{\Gamma}}{M}{\rvTyEnd}$}
        \RightLabel{\rlabel{\textsc{T-Close}}{rule:rv-ty-close}}
        \UIC{$\rvSeq{\ty{\Gamma}}{\rvClose{M}}{\rvTyOpt{\rvTyUnit}}$}
      \end{prooftree*}
      
      \begin{prooftree*}
        \AXC{$\rvSeq{\ty{\Gamma}}{M}{A}$}
        \AXC{$\rvSeq{\ty{\Delta}}{N}{\rvTySend{A}{S}}$}
        \RightLabel{\rlabel{\textsc{T-Send}}{rule:rv-ty-send}}
        \BIC{$\rvSeq{\ty{\Gamma}\st{,}\;\ty{\Delta}}{\rvSend{M}{N}}{S}$}
      \end{prooftree*}
      \begin{prooftree*}
        \AXC{$\rvSeq{\ty{\Gamma}}{M}{\rvTyRecv{A}{S}}$}
        \RightLabel{\rlabel{\textsc{T-Recv}}{rule:rv-ty-recv}}
        \UIC{$\rvSeq{\ty{\Gamma}}{\rvRecv{M}}{\rvTyOpt{\rvPair{A}{S}}}$}
      \end{prooftree*}
    \end{center}
    
    \header{Duality}{\rvDual{S}}%
    \vspace*{0.5\baselineskip}%
    \begin{center}
      \(
      \rvDual{\rvTySend{A}{S}} = \rvTyRecv{A}{\rvDual{S}}
      \quad
      \rvDual{\rvTyRecv{A}{S}} = \rvTySend{A}{\rvDual{S}}
      \quad
      \rvDual{\rvTyEnd} = \rvTyEnd
      \)
    \end{center}
  \end{highlight}\end{mdframed}
\caption{Rusty Variation, static typing.}
\label{fig:rv-static-typing}
\end{figure*}

%%% Local Variables:
%%% TeX-master: "main"
%%% End:
\begin{figure*}
  \begin{mdframed}
    \centering
    \(
    \begin{array}{llrl}
      \text{Channels}
      &\tm{a}, \tm{b}, \tm{c}
      \\
      \text{Terms}
      &\tm{L}, \tm{M}, \tm{N}
      &::= & \dots
             \sep \tm{a}
      \\
      \text{Values}
      &\tm{U}, \tm{V}, \tm{W}
      &::= & \tm{a}
             \sep \rvAbs[A]{x}{M}
             \sep \rvUnit
             \sep \rvPair{V}{W}
             \sep \rvInl{V}
             \sep \rvInr{V}
      \\
      \text{Memory}
      &\tm{\rvMem{M}}
      &::= & \rvMemEmp
             \sep \rvMemData{V}{c}
      \\
      \text{Heaps}
      &\tm{\rvHeap{H}}
      &::= & \rvHeapEmp
             \sep \rvHeapData{\rvHeap{H}}{a}{b}{\rvMem{M}}
             \sep \rvHeapDisc{\rvHeap{H}}{a}{b}
      \\
      \text{Thread flags}
      &\tm{\phi}
      &::= & \rvFlagMain
             \sep \rvFlagChild
      \\
      \text{Threads}
      &\tm{T}
      &::= & \rvThread{\phi}{M}
             \sep \rvPanic
      \\
      \text{Configurations}
      &\rvConf{C}, \rvConf{D}, \rvConf{E}
      &    & \text{multiset of threads}
      \\
      \text{Evaluation contexts}
      &\tm{E}
      &::= & \rvHole
             \sep \rvApp{E}{M}
             \sep \rvApp{V}{E}
             \sep \rvLetUnit{E}{M}
             \sep \rvPair{E}{M}
             \sep \rvPair{V}{E}
             \sep \rvLetPair{x}{y}{E}{M}
      \\
      &
      &\mid& \rvInl{E}
             \sep \rvInr{E}
             \sep \rvCaseSum{E}{x}{M}{y}{N}
      \\
      &
      &\mid& \rvFork{E}
             \sep \rvSend{E}{M}
             \sep \rvSend{V}{E}
             \sep \rvRecv{E}
             \sep \rvClose{E}
      \\
      \text{Thread contexts}
      &\tm{F}
      &::= & \rvThread{\phi}{E}
    \end{array}
    \)
  \end{mdframed}
  \caption{Rusty Variation, runtime syntax.}
  \label{fig:rv-runtime}
\end{figure*}

%%% Local Variables:
%%% TeX-master: "main"
%%% End:

\begin{figure*}
  \begin{mdframed}
    \centering
    \[\!\!%
      \setlength{\arraycolsep}{4pt}%
      \begin{array}{llcl}
        \textsc{E-Lam}
        & {\rvApp{\rvAbs[A]{x}{M}}{V}}
        & \rvRedArrPure
        & {\rvSub{M}{x}{V}}
        \\
        \textsc{E-Unit}
        & {\rvLetUnit{\rvUnit}{M}}
        & \rvRedArrPure
        & {\rvVar{M}}
        \\
        \textsc{E-Pair}
        & {\rvLetPair{x}{y}{\rvPair{V}{W}}{M}}
        & \rvRedArrPure
        & {\rvSub{M}{\tm{x}\st{,}\;\tm{y}}{\tm{V}\st{,}\;\tm{W}}}
        \\
        \textsc{E-Inl}
        & {\rvCaseSum{\rvInl{V}}{x}{M}{y}{N}}
        & \rvRedArrPure
        & {\rvSub{M}{x}{V}}
        \\
        \textsc{E-Inr}
        & {\rvCaseSum{\rvInr{V}}{x}{M}{y}{N}}
        & \rvRedArrPure
        & {\rvSub{N}{y}{V}}
        \\
        \textsc{E-Some}
        & {\rvCaseOpt{\rvSome{V}}{x}{M}{N}}
        & \rvRedArrPure
        & {\rvSub{M}{x}{V}}
        \\
        \textsc{E-None}
        & {\rvCaseOpt{\rvNone}{x}{M}{N}}
        & \rvRedArrPure
        & {\rvVar{N}}
        \\
        \textsc{E-Lift}
        & {\rvPlug{E}{M}}
        & \rvRedArrPure
        & {\rvPlug{E}{M'}}\st{,}
          \quad\text{if}\;\rvRedPure{M}{M'}
      \end{array}
    \]
    {
      \setstretch{2.0}
      \begin{prooftree}
        \AXC{\tm{a} is fresh}
        \RightLabel{\textsc{E-Fork}}
        \UIC{$
          \rvRedTup{\rvHeap{H}}{\rvPlug{F}{\rvFork{\rvAbs[S]{x}{M}}}}
          \rvRedArr
          \rvRedTup{\rvHeapData{\rvHeap{H}}{a}{\rvMemEmp}}{\rvPlug{F}{a}\st{,}\;\rvChild{\rvSub{M}{x}{a}}}$}
      \end{prooftree}
      \begin{prooftree}
        \AXC{\tm{b} is fresh}
        \RightLabel{E-Send}
        \UIC{$
          \rvRedTup{\rvHeapData{\rvHeap{H}}{a}{\rvMemEmp}}{\rvPlug{F}{\rvSend{U}{a}}}
          \rvRedArr
          \rvRedTup{\rvHeapData{\rvHeapData{\rvHeap{H}}{a}{\rvMemData{U}{b}}}{b}{\rvMemEmp}}{\rvPlug{F}{b}}$}
      \end{prooftree}
      \begin{prooftree}
        \AXC{}
        \RightLabel{\textsc{E-Recv}}
        \UIC{$
          \rvRedTup{\rvHeapData{\rvHeap{H}}{a}{\rvMemData{U}{b}}}{\rvPlug{F}{\rvRecv{a}}}
          \rvRedArr
          \rvRedTup{\rvHeap{H}}{\rvPlug{F}{\rvSome{\rvPair{U}{b}}}}$}
      \end{prooftree}
      \begin{prooftree}
        \AXC{}
        \RightLabel{\textsc{E-Close}}
        \UIC{$
          \rvRedTup
          {\rvHeapData{\rvHeap{H}}{a}{\rvMemEmp}}
          {\rvPlug{F}{\rvClose{a}}\st{,}\;\rvPlug{F'}{\rvClose{a}}}
          \rvRedArr
          \rvRedTup{\rvHeap{H}}{\rvPlug{F}{\rvSome{\rvUnit}}\st{,}\;\rvPlug{F'}{\rvSome{\rvUnit}}}$}
      \end{prooftree}
      \begin{center}
        \begin{prooftree*}
          \AXC{$\rvArc{a}{\rvHeap{H}}{\rvConf{C}}\le{1}$}
          \RightLabel{\textsc{E-Zap}}
          \UIC{$
            \rvRedTup[\rvConf{C}]{\rvHeapData{\rvHeap{H}}{a}{\rvMem{M}}}{}
            \rvRedArr
            \rvRedTup[\rvConf{C}]{\rvHeapDisc{\rvHeap{H}}{a}}{}$}
        \end{prooftree*}%
        \begin{prooftree*}
          \AXC{$\rvArc{a}{\rvHeap{H}}{\rvConf{C}}={0}$}
          \RightLabel{\textsc{E-GC}}
          \UIC{$
            \rvRedTup[\rvConf{C}]{\rvHeapDisc{\rvHeap{H}}{a}}{}
            \rvRedArr
            \rvRedTup[\rvConf{C}]{\rvHeap{H}}{}$}
        \end{prooftree*}
      \end{center}
      \begin{prooftree}
        \AXC{}
        \RightLabel{\textsc{E-CloseZap}}
        \UIC{$
          \rvRedTup{\rvHeapDisc{\rvHeap{H}}{a}}{\rvPlug{F}{\rvClose{a}}}
          \rvRedArr
          \rvRedTup{\rvHeapDisc{\rvHeap{H}}{a}}{\rvPlug{F}{\rvNone}}$}
      \end{prooftree}
      \begin{prooftree}
        \AXC{}
        \RightLabel{\textsc{E-RecvZap}}
        \UIC{$
          \rvRedTup{\rvHeapDisc{\rvHeap{H}}{a}}{\rvPlug{F}{\rvRecv{a}}}
          \rvRedArr
          \rvRedTup{\rvHeapDisc{\rvHeap{H}}{a}}{\rvPlug{F}{\rvNone}}$}
      \end{prooftree}
      \begin{center}
        \begin{prooftree*}
          \AXC{}
          \RightLabel{\textsc{E-HaltChild}}
          \UIC{$
            \rvRedTup[\rvConf{C}]{\rvHeap{H}}{\rvChild{\rvVar{V}}}
            \rvRedArr
            \rvRedTup[\rvConf{C}]{\rvHeap{H}}{}$}
        \end{prooftree*}%
        \begin{prooftree*}
          \AXC{}
          \RightLabel{\textsc{E-HaltMain}}
          \UIC{$
            \rvRedTup[\rvConf{C}]{\rvHeap{H}}{\rvMain{\rvNone}}
            \rvRedArr
            \rvRedTup[\rvConf{C}]{\rvHeap{H}}{\rvPanic}$}
        \end{prooftree*}
      \end{center}
      \begin{prooftree}
        \AXC{$\rvRedPure{M}{M'}$}
        \RightLabel{\textsc{E-LiftM}}
        \UIC{$
          \rvRedTup{\rvHeap{H}}{\rvThread{\phi}{M}}
          \rvRedArr
          \rvRedTup{\rvHeap{H}}{\rvThread{\phi}{M'}}$}
      \end{prooftree}
    }
  \end{mdframed}

  \caption{Rusty Variation, reduction semantics.}
  \label{fig:rv-reduction}
\end{figure*}


\subsection{Linear and Affine EGV}
EGV is a linear language, in the sense that it does not allow the implicit discarding of names, \ie implicit weakening. However, it does support \emph{explicit} weakening, in the form of \ref{rule:egv-ty-cancel}. RV supports implicit weakening. Our first step in investigating EGV's relation to RV is construct \emph{affine} EGV, a variant of EGV which \emph{does} support implicit weakening.

We make three changes to EGV (hence linear EGV or \linearEGV) to obtain affine EGV (\affineEGV):
\begin{enumerate}
\item 
  Remove the $\gvCancel{x}$ construct and the corresponding typing and reduction rules.
\item
  Add a typing rule to allow implicit weakening:
  \begin{prooftree}
    \AXC{$\gvSeq{\ty{\Gamma}}{M}{A}$}
    \RightLabel{\rlabel{\textsc{T-Weak}}{rule:egv-ty-weak}}
    \UIC{$\gvSeq{\ty{\Gamma}\st{,}\;\tmty{x}{B}}{M}{A}$}
  \end{prooftree}
\item
  Add a garbage collection reduction, where $\gvArc{\gvVar{a}}{\gvConf{C}}$ counts the number of occurrences of $\gvVar{a}$ in $\gvConf{C}$:
  \begin{center}
    \(
    \rlabel{\textsc{E-Drop}}{rule:egv-red-drop}
    \quad
    {\gvRes{a}{\gvConf{C}}}
    \gvRedArr
    {\gvRes{a}{\gvPPar{\gvZap{a}}{\gvConf{C}}}}
    \quad
    \text{if} \; \gvArc{\gvVar{a}}{\gvConf{C}} = 1
    \)
  \end{center}
\end{enumerate}
We can translate programs in \linearEGV to \affineEGV by replacing all occurrences of $\gvCancel{x}$ with $\gvUnit$. We name this translation $\lta{}$. This translation preserves typing, witnessed by the fact that \ref{rule:egv-ty-cancel} is derivable in affine EGV:%
\begin{prooftree}
  \AXC{}
  \RightLabel{\ref{rule:egv-ty-unit}}
  \UIC{$\gvSeq{\emptyenv}{\gvUnit}{\gvTyUnit}$}
  \RightLabel{$\ref{rule:egv-ty-weak}^\star$}
  \UIC{$\gvSeq{\ty{\Gamma}}{\gvUnit}{\gvTyUnit}$}
\end{prooftree}
We have a \emph{strict} operational correspondence between \linearEGV and \affineEGV along $\lta{}$, \ie the following diagrams commute:
\[
  \begin{tikzcd}
    \linearEGV
    \rar{\gvRedArrPure}
    \dar{\lta{}}
    &
    \linearEGV
    \dar{\lta{}}
    \\
    \affineEGV
    \rar{\gvRedArrPure}
    &
    \affineEGV
  \end{tikzcd}
  \quad
  \begin{tikzcd}
    \linearEGV
    \rar{\gvRedArr}
    \dar{\lta{}}
    &
    \linearEGV
    \dar{\lta{}}
    \\
    \affineEGV
    \rar{\gvRedArr}
    &
    \affineEGV
  \end{tikzcd}
\]
On the left, \linearEGV and \affineEGV denote the respective terms. This diagram commutes as the term languages of the two systems are identical. On the right, \linearEGV and \affineEGV denote the respective configurations up to structural congruence. To show that this diagram commutes, we observe that each application of \ref{rule:egv-red-cancel} can be replaced with an application of \ref{rule:egv-red-drop} after the translation $\lta{}$ has been applied, and vice versa.

In the other direction, we can translate programs in \affineEGV to \linearEGV. We name this translation $\atl{}$. This translation needs access to the typing environment of the program, as dropped names do not necessarily show up in the term. Therefore, we choose to define the translation $\atl{}$ on \emph{typing derivations}, by replacing each application of \ref{rule:egv-ty-weak} with a derived version:%
\begin{prooftree}
  \AXC{}
  \RightLabel{\ref{rule:egv-ty-var}}
  \UIC{$\gvSeq{\tmty{x}{B}}{x}{B}$}
  \RightLabel{\ref{rule:egv-ty-cancel}}
  \UIC{$\gvSeq{\tmty{x}{B}}{\gvCancel{x}}{B}$}
  \AXC{$\gvSeq{\ty{\Gamma}}{M}{A}$}
  \RightLabel{\ref{rule:egv-ty-letunit}}
  \BIC{$\gvSeq{\ty{\Gamma}\st{,}\;\tmty{x}{B}}{\gvLetUnit{\gvCancel{x}}{M}}{A}$}
\end{prooftree}
We have a strict operational correspondence between \affineEGV and \linearEGV along $\atl{}$, \ie the following diagrams commute:
\[
  \begin{tikzcd}
    \affineEGV
    \rar{\gvRedArrPure}
    \dar{\atl{}}
    &
    \affineEGV
    \dar{\atl{}}
    \\
    \linearEGV
    \rar{\gvRedArrPure}
    &
    \linearEGV
  \end{tikzcd}
  \quad
  \begin{tikzcd}
    \affineEGV
    \rar{\gvRedArr}
    \dar{\atl{}}
    &
    \affineEGV
    \dar{\atl{}}
    \\
    \linearEGV
    \rar{\gvRedArr}
    &
    \linearEGV
  \end{tikzcd}
\]
The proof is similar to the proof for $\lta{}$, except that it is on the level of typing derivations. From these two operational correspondences, we can infer that \affineEGV preserves the metatheory of \linearEGV: it satisfies preservation and progress, is deadlock free, confluent, and terminating.

There exist variants of $\atl{}$ on the level of terms. However, these are less well-behaved. In defining $\atl{}$ on terms, there is a certain freedom in deciding where to place the explicit cancellation, corresponding to the commuting conversions for weakening. Furthermore, composing $\lta{}$ and $\atl{}$ on the term-level, there is no way to recover dropped free names, unless these are explicitly provided to the hypothetical $\atl{}$.

\subsection{Exceptions and the Option Monad}
EGV supports exceptions. Rust does not support exceptions. Our second step in investigating EGV's relation to RV is to construct affine asynchronous \emph{GV} (\affineAGV), a variant of \affineEGV which does not support exceptions.

We make four changes to \affineEGV to obtain \affineAGV:
\begin{enumerate}
\item
  Remove the $\gvRaise$ and $\gvTry{L}{x}{M}{N}$ constructs and their corresponding typing and reduction rules.
\item
  Add syntactic sugar for an option type, similar to the option type in RV~(\cref{fig:rv-syntactic-sugar}):
  \\
  \begin{minipage}{1.0\linewidth}
    \[%
      \begin{array}{l}%
        \!\!\!%
        \begin{array}{lcl}%
          \gvTyOpt{A} &::=& \gvTySum{A}{\gvTyUnit}
          \\
          \gvSome{x} &::=& \gvInl{x}
          \\
          \gvNone &::=& \gvInr{\gvUnit}
        \end{array}
        \\
        \gvCaseOpt{L}{x}{M}{N} ::=
        \\
        \quad\gvCaseSum{L}{x}{M}{y}{\gvLetUnit{y}{N}}
      \end{array}
    \]
  \end{minipage}
\item
  Change \ref{rule:egv-ty-fork}, \ref{rule:egv-ty-close}, \ref{rule:egv-ty-recv}, \ref{rule:egv-ty-main}, and \ref{rule:egv-ty-child} to mark the possibility for failure, similar to the equivalent rules in RV~(\cref{fig:rv-static-typing}):
  \begin{center}
    \begin{prooftree}
      \AXC{$\gvSeq{\ty{\Gamma}}{M}{\gvTyFun{S}{\gvTyOpt{\gvTyUnit}}}$}
      \RightLabel{\rlabel{\textsc{T-Fork}}{rule:egv-ty-fork-err}}
      \UIC{$\gvSeq{\ty{\Gamma}}{\gvFork{M}}{\gvDual{S}}$}
    \end{prooftree}    
    \begin{prooftree}
      \AXC{$\gvSeq{\ty{\Gamma}}{M}{\gvTyEnd}$}
      \RightLabel{\rlabel{\textsc{T-Close}}{rule:egv-ty-close-err}}
      \UIC{$\gvSeq{\ty{\Gamma}}{\gvClose{M}}{\gvTyOpt{\gvUnit}}$}
    \end{prooftree}
    \begin{prooftree}
      \AXC{$\gvSeq{\ty{\Gamma}}{M}{\gvTyRecv{A}{S}}$}
      \RightLabel{\rlabel{\textsc{T-Recv}}{rule:egv-ty-recv-err}}
      \UIC{$\gvSeq{\ty{\Gamma}}{\gvRecv{M}}{\gvTyOpt{\gvPair{A}{S}}}$}
    \end{prooftree}
    \begin{prooftree*}
      \AXC{$\gvSeq{\ty{\Gamma}}{M}{\gvTyOpt{A}}$}
      \RightLabel{\rlabel{\textsc{T-Main}}{rule:egv-ty-main}}
      \UIC{$\gvCSeq{\ty{\Gamma}}{\emptyenv}{\gvFlagMain}{\gvMain{M}}$}
    \end{prooftree*}%
    \begin{prooftree*}
      \AXC{$\gvSeq{\ty{\Gamma}}{M}{\gvTyOpt{\gvTyUnit}}$}
      \RightLabel{\rlabel{\textsc{T-Child}}{rule:egv-ty-child}}
      \UIC{$\gvCSeq{\ty{\Gamma}}{\emptyenv}{\gvFlagChild}{\gvChild{M}}$}
    \end{prooftree*}
  \end{center}
\item
  Add reductions to deal with top-level values of the option type:
  \begin{center}
    \begin{prooftree}
      \AXC{}
      \RightLabel{\rlabel{\textsc{E-HaltChild}}{rule:egv-red-haltchild}}
      \UIC{$
        \gvPar{\gvConf{C}}{\gvChild{V}}
        \gvRedArr
        \gvConf{C}$}
    \end{prooftree}%
    \begin{prooftree}
      \AXC{}
      \RightLabel{\rlabel{\textsc{E-HaltMain}}{rule:egv-red-haltmain}}
      \UIC{$
        \gvMain{\gvNone}
        \gvRedArr
        \gvHalt$}
    \end{prooftree}
  \end{center}
\end{enumerate}
We can translate programs in \affineEGV to \affineAGV following Filinski~\cite{filinski1994}. We name this translation $\atm{}$. The definition of $\atm{}$ is given in \cref{fig:affine-egv-to-agv}.

\subsection{Channels and Shared State}

\bibliographystyle{ACM-reference-format}
\bibliography{main}

\end{document}