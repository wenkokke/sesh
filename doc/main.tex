\documentclass[a4paper,parskip=never,UKenglish]{scrartcl}
\usepackage{textcomp}
\usepackage{textgreek}
\usepackage{doi}
\usepackage[all]{foreign}

% compact enums
\usepackage{enumitem}
\setlist[enumerate]{nosep}

% red squiggly
\usepackage{ulem}
\normalem%
\makeatletter
\def\uwave{\bgroup \markoverwith{\lower3.5\p@\hbox{\sixly \textcolor{red}{\char58}}}\ULon}
\font\sixly=lasy6 % does not re-load if already loaded, so no memory problem.
\makeatother

% load listings-rust
\usepackage{inconsolata}
\usepackage{listings}
\usepackage{listings-rust}
\lstset{
  language=Rust,
  style=colouredRust,
  basicstyle=\ttfamily\small,
  columns=fullflexible,
  xleftmargin=\parindent,
  moredelim=[is][\uwave]{~}{~}
}

% load cleverref
\usepackage{cleveref}
\addbibresource{main.bib}
\title{Rusty Variation}
\subtitle{Affine Session Types for Rust}
\author{%
  Wen Kokke\protect\\
  Supervised by Philip Wadler and Ian Stark\protect\\
  Graduate student (8450698)}
\affil{
  LFCS, University of Edinburgh\protect\\
  Informatics Forum, 10 Crichton St, Edinburgh EH8 9AB, UK\protect\\
  \email{wen.kokke@ed.ac.uk}}
\date{}
\begin{document}
\maketitle
\begin{abstract}
  Rusty Variation (RV) is a library for session-typed communication in Rust, which offers strong compile-time correctness guarantees. Programs written using RV are guaranteed to adhere to a specified protocol, and are guaranteed to be free from dead- and livelocks. RV is an implementation of Exceptional GV, a session-typed functional language with exceptions and session cancellation.
\end{abstract}

\section{Problem and Motivation}

\section{Background and Related Work}
\subsection{Session Types for Rust}
\textcite{jespersen2015}

\subsection{Exceptional GV}
\textcite{fowler2018}

\section{Approach and Uniqueness}
\begin{lstlisting}[language=Rust, style=boxed]
  use std::marker;
  use rusty_variation::*;

  type NegServer  = Receive<i32, Send<i32, End>>;
  type AddServer  = Receive<i32, Receive<i32, Send<i32, End>>>;
  type CalcServer = Receive<CalcOp, End>;
  enum CalcOp { Neg(NegServer), Add(AddServer) }
  
  #[test]
  fn calc_server_works() {  
    assert!(|| -> Result<i32, Box<Error>> {
      
      // Fork a calculator server.
      let s = fork!(move |s: CalcServer| {
        offer!(s, {
          CalcOp::Neg(s) => {
            let (x, s) = receive(s)?;
            let s = send(-x, s)?;
            close(s)
          },
          CalcOp::Add(s) => {
            let (x, s) = receive(s)?;
            let (y, s) = receive(s)?;
            let s = send(x + y, s)?;
            close(s)
          },
        })
      });
         
      // Send it the numbers. 
      let s = select!(CalcOp::Add, s)?;
      let s = send(4, s)?;
      let s = send(5, s)?;
        
      // Receive the answer and close the channel.
      let (z, s) = receive(s)?;
      close(s)?;
        
      // Check the answer.
      assert_eq!(z, 9);
        
      Ok(())

  }().is_ok());  
\end{lstlisting}

\section{Results and Contributions}

\clearpage
\printbibliography
\end{document}